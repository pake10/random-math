\documentclass[12pt,finnish]{exam}
\usepackage{graphicx}
\graphicspath{ {} }
\usepackage[utf8]{inputenc}
\usepackage{graphicx}
\usepackage[rightcaption]{sidecap}
\usepackage{wrapfig}
\usepackage{babel}
\usepackage{enumitem}
\usepackage{amsmath}
\usepackage{amssymb}
\usepackage[T1]{fontenc}
\usepackage{tikz}
\setlist[enumerate,1]{leftmargin=*}

\begin{document}
 \section*{Lineaarialgebran jatkoa}
 
\vspace{5mm}
 
\begin{questions}
\bfseries
\question
\mdseries
\textbf{a)} Oletetaan, että vektoreille \(\overline{v}, \overline{w}\in \mathbb{R}^n\) pätee \(||\overline{v}||=3\), \(||\overline{w}||=4\) ja \(\overline{v}\cdot \overline{w}=-3\). Määritä\\ \(\text{    }||2\overline{v}-\overline{w}||\).\\
 \\
\textbf{b)} Määritä matriisin \(A=\begin{bmatrix} 1 & 4 \\ 2 & 1 \end{bmatrix}\) kaikki ominaisarvot ja ominaisvektorit.

\bfseries
\question
\mdseries
Oletetaan, että \(\overline{v}, \overline{w}, \overline{u}\in \mathbb{R}^n\) ja \(\overline{w}\neq \overline{0}\). Oletetaan lisäksi, että vektori \(\overline{u}\) on yhdensuuntainen vektorin \(\overline{w}\) kanssa. Osoita, että \(\text{proj}_{\overline{w}}(\overline{v})=\text{proj}_{\overline{u}}({\overline{v}})\). Tulkitse tulos geometrisesti avaruudessa \(\mathbb{R}^2\).


\bfseries
\question
\mdseries
Merkitään \(\overline{w}_1=(1, 2, 0)\), \(\overline{w}_2=(1, 1, -1)\) ja \(\overline{w}_3=(1, 4, 2)\).\\
 \\
\textbf{a)} Kuuluuko vektori \(\overline{v}=(1, 1, 0)\) aliavaruuteen \(\text{span}\left (\overline{w}_1, \overline{w}_2, \overline{w}_3\right )\)?\\
\textbf{b)} Muodostavatko vektorit \(\overline{w}_1\), \(\overline{w}_2\), \(\overline{w}_3\) avaruuden \(\mathbb{R}^3\) kannan?

\bfseries
\question
\mdseries
Olkoot \(a\), \(b\) ja \(c\) positiivisia reaalilukuja. Tetraedrin kolme kärkeä ovat koordinaattiakseleiden pisteissä \((a, 0, 0)\), \((0, b, 0)\) ja \((0, 0, c)\), ja neljäs kärki on origossa \((0, 0, 0)\). Kärkien vastaisten tetraedrin tahkojen pinta-aloja merkitään samassa järjestyksessä kirjaimilla \(A\), \(B\), \(C\) ja \(D\), jossa \(D\) tarkoittaa origon vastaisen tahkon pinta-alaa.\\
 \\
Määritä \(D\) ristitulon avulla ja osoita, että $$A^2+B^2+C^2=D^2.$$
 \\
\centering
\includegraphics[width=0.5\textwidth]{kuva.png}

\end{questions}
\thispagestyle{empty}


\end{document}

