\documentclass[12pt,finnish]{exam}
\usepackage[utf8]{inputenc}
\usepackage{graphicx}
\usepackage[rightcaption]{sidecap}
\usepackage{wrapfig}
\usepackage{babel}
\usepackage{enumitem}
\usepackage{amsmath}
\usepackage{amssymb}
\usepackage[T1]{fontenc}
\usepackage{lmodern}
\usepackage{tikz}
\setlist[enumerate,1]{leftmargin=*}

\begin{document}
 \section*{Lukuteoria ja todistaminen}
 
\vspace{5mm}
 
\begin{questions}
\bfseries
\question
\mdseries
\begin{enumerate}[label=\textbf{\alph*)}]
\item Olkoot $p$ ja $q$ loogisia lauseita. Onko lause $\neg (p \lor \neg p) \Leftrightarrow (p \land q)$ kontradiktio?
\item Olkoon \(M(x):\) "$x$ on ihminen". Muunna luonnolliselle kielelle lause \(\nexists x:M(x)\).
\end{enumerate}
\vspace{1ex}


\bfseries
\question
\mdseries
Osoita: Jos mielivaltainen luku on jaollinen luvulla 3, niin myös sen numeroiden summa on kolmella jaollinen.

\bfseries
\question
\mdseries
Olkoon \(4^n-1\) alkuluku, \(n\in \mathbb{N}\). Todista epäsuoraa todistusta käyttämällä, että tällöin \(n\) on pariton.

\vspace{1ex}


\bfseries
\question
\mdseries
\begin{enumerate}[label=\textbf{\alph*)}]
\item Määritä luvun $79^n+67^{2n}$ viimeinen numero, kun \(n\) on kokonaisluku.

\item Osoita induktiolla, että luku \(6^n-1\), \(n\in \mathbb{N}\), on aina jaollinen luvulla 5.
\end{enumerate}

\bfseries
\question
\mdseries
\begin{enumerate}[label=\textbf{\alph*)}]
\item Määritä suoralta \(86x+64,\!5y=43\) piste, jonka koordinaatit ovat kokonaislukuja.

\item Määritä lukujen 16 360 ja 8 265 suurin yhteinen tekijä alkulukuhajotelmien avulla.
\end{enumerate}

\end{questions}
\textbf{Bonus: } Kuinka monta numeroa on luvussa $x=2017^{2017}$? (Hyödynnä logaritmeja.)
\thispagestyle{empty}


\end{document}

