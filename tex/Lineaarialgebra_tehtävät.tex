\documentclass[12pt,finnish]{exam}
\usepackage[utf8]{inputenc}
\usepackage{graphicx}
\usepackage[rightcaption]{sidecap}
\usepackage{wrapfig}
\usepackage{babel}
\usepackage{enumitem}
\usepackage{amsmath}
\usepackage{amssymb}
\usepackage[T1]{fontenc}
\usepackage{tikz}
\setlist[enumerate,1]{leftmargin=*}

\begin{document}
 \section*{Lineaarialgebra}
 
\vspace{5mm}
 
\begin{questions}
\bfseries
\question
\mdseries
Määritellään avaruudessa \(\mathbb{R}^3\) vektorit \(A=\begin{bmatrix}
       1         \\
       2         \\
       3
     \end{bmatrix}\)
     ja \(B=\begin{bmatrix}
       -1         \\
       3         \\
       5
     \end{bmatrix}\).

\begin{enumerate}[label=\textbf{\alph*)}]
\item Laske vektoreiden sisätulo.
\item Vektorit virittävät origon kautta kulkevan tason. Määrää tason yhtälö muodossa \(ax+by+cz+d=0\).
\end{enumerate}

\bfseries
\question
\mdseries
Olkoon \(A=\begin{bmatrix}
       1&1&2         \\
       k&1&1         \\
       1&k&2
     \end{bmatrix}\), \(k\in \mathbb{R}\). Määritä \(\det A\). Millä \(k\):n arvoilla matriisi \(A\) on kääntyvä?

\bfseries
\question
\mdseries  
Matriisi \(Q\) on \emph{ortogonaalinen}, jos sen transpoosimatriisi on sama kuin sen käänteismatriisi eli \(QQ^{\text{T}}=I\). Osoita, että jokainen ortogonaalinen \(2\times 2\)-matriisi voidaan kirjoittaa joko muotoon $$\begin{bmatrix} \cos \varphi & \sin \varphi \\ -\sin \varphi & \cos \varphi \end{bmatrix} \text{ tai } \begin{bmatrix} \cos \varphi & \sin \varphi \\ \sin \varphi & -\cos \varphi \end{bmatrix},$$ missä \(0\leq \varphi \leq 2\pi\).

\bfseries
\question
\mdseries
Ratkaise Gaussin-Jordanin eliminointimenetelmän avulla yhtälöryhmä \(\begin{cases} 2x-2y+4z&=0 \\ x+z&=0 \\ 2x-y+3z&=0. \end{cases}\)
\end{questions}
\thispagestyle{empty}


\end{document}

