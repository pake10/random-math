\documentclass[12pt,finnish]{exam}
\usepackage[utf8]{inputenc}
\usepackage{graphicx}
\usepackage[rightcaption]{sidecap}
\usepackage{wrapfig}
\usepackage{babel}
\usepackage{enumitem}
\usepackage{amsmath}
\usepackage{amssymb}
\usepackage[T1]{fontenc}
\usepackage{tikz}
\setlist[enumerate,1]{leftmargin=*}

\begin{document}
 \section*{Numeerisia ja algebrallisia menetelmiä}
 
\vspace{5mm}
 
\begin{questions}

\bfseries
\question
\mdseries
Määrää polynomin \(x^3-x^2+2x-2\) kaikkien nollakohtien tarkat arvot.

\vspace{1ex}

\bfseries
\question
\mdseries
Tarkastellaan yhtälöä \(\ln x-x=-2\).
\begin{enumerate}[label=\textbf{\alph*)}]
\item Osoita, että yhtälöllä on tasan yksi juuri välillä \([2,4]\).
\item Määritä tämä juuri kolmen desimaalin tarkkuudella käyttäen kiintopistemenetelmää.
\end{enumerate}

\bfseries
\question
\mdseries
Italialainen Fibonacci laski vuonna 1225 yhtälön \(x^3+2x^2+10x-20=0\) juurelle likiarvon \(x\approx 1,\! 368808108\). \emph{(yo 15k)}
\begin{enumerate}[label=\textbf{\alph*)}]
\item Osoita, että yhtälöllä on täsmälleen yksi juuri reaalilukujen joukossa.
\item {Kuinka mones Newtonin menetelmän iterointikierros tuottaa ensimmäisen kerran samat yhdeksän desimaalia kuin Fibonaccin likiarvossa, kun alkuarvona on \(x_0=1\)?}
\end{enumerate}

\vspace{1ex}

\bfseries
\question
\mdseries
Veden virtausnopeutta putkessa mitattiin tunnin välein. Mittaustulokset on esitetty oheisessa taulukossa.
\begin{enumerate}[label=\textbf{\alph*)}]
\item Arvioi puolisuunnikassäännön avulla, kuinka monta litraa vettä virtasi mittauspisteen ohi mittauksen aikana.

\item Oletetaan, että virtausnopeuden kuvaaja on polynomi. Arvioi a-kohdan tuloksen tarkkuutta kuvion avulla.

\end{enumerate}


\begin{center}
\begin{tabular}{ |c|c|c|c|c|c|c|c| } 
 \hline
 \(t\) (h) & 0 & 1 & 2 & 3 & 4 & 5 & 6 \\ 
 \hline
 \(Q\) (l/h) & 0,\! 50 & 1,\! 00 & 0,\! 60 & 0,\! 50 & 0,\! 40 & 0,\! 80 & 1,\! 00 \\ 
 \hline
\end{tabular}
\end{center}

\bfseries
\question
\mdseries
Kappale liikkuu \((t,x)\)-koordinaatistossa pitkin käyrää \(\sin (2t)\) välillä \([0,1]\). Aika on ilmoitettu sekunteina ja paikka metreinä.
\begin{enumerate}[label=\textbf{\alph*)}]
\item Arvioi kappaleen kulkemaa matkaa Simpsonin säännöllä, kun osavälejä on \(n=2\).
\item Miten arvion tarkkuus muuttuu, kun \(n \to \infty\)? Piirrä kuvio.
\end{enumerate}

\end{questions}
\thispagestyle{empty}


\end{document}

