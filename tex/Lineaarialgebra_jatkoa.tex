\documentclass{article}
\usepackage[utf8]{inputenc}
\date{}

\title{\textbf{Lineaarialgebran jatkoa}}
\author{Jimi Käyrä}

\usepackage{graphicx}
\usepackage{amsmath}
\usepackage{amssymb}
\usepackage{tikz}
\usepackage{icomma}
\usepackage{pgfplots}
\usepackage{array}
\usepackage{eurosym}
\usepackage{arydshln}
\usepackage{relsize}
\usepackage[version=4]{mhchem}
\pgfplotsset{compat=1.8}
\usepackage{mathtools}
\usepackage{bm}
\usetikzlibrary{decorations.pathreplacing}
\usetikzlibrary{matrix}
\renewcommand{\contentsname}{Sisällys}
\begin{document}
\begin{titlepage}
\maketitle
\tableofcontents
\end{titlepage}

\section{Avaruuden \(\mathbb{R}^n\) vektorit}
Oletetaan, että \(n \in {1, 2, 3, ...}\). Avaruuden \(\mathbb{R}^n\) alkiot ovat jonoja, joissa on \(n\) reaalilukua eli $$\mathbb{R}^n=\{(v_1, v_2, ..., v_n) \text{ }|\text{ } v_1, v_2, ..., v_n \in \mathbb{R}\}.$$ Avaruuden \(\mathbb{R}^n\) alkioita kutsutaan \textbf{vektoreiksi}.\\
 \\
Jos \(u_1, u_2, ..., u_n \in \mathbb{R}\), niin \(\overline{u}=(u_1, u_2, ..., u_n)\) on avaruuden \(\mathbb{R}^n\) vektori ja sanotaan, että \(u_1, u_2, ..., u_n\) ovat vektorin \(\overline{u}\) \textbf{komponentit}.\\
 \\
Vektori \(\overline{w}\) on vektoreiden \(\overline{v}_1, \overline{v}_2, ..., \overline{v}_k\) \textbf{lineaarikombinaatio}, jos on olemassa sellaiset reaaliluvut \(a_1, a_2, ..., a_k\), että $$\overline{w}=a_1\overline{v}_1+a_2\overline{v}_2+ ... + a_k \overline{v}_k.$$
 \\
\textbf{Esim.} Olkoot \(\overline{v}_1=(1, 1)\), \(\overline{v}_2=(-1, 2)\) ja \(\overline{w}=(5, -1)\). Osoita, että vektori \(\overline{w}\) on vektoreiden \(\overline{v}_1\) ja \(\overline{v}_2\) lineaarikombinaatio.

\newpage
\section{Vektoreiden virittämä aliavaruus}
Vektoreiden \(\overline{v}_1, \overline{v}_2, ..., \overline{v}_k \in \mathbb{R}\) virittämä \textbf{aliavaruus} tarkoittaa kyseisten vektoreiden kaikkien lineaarikombinaatioiden joukkoa eli $$W=\text{span}(\overline{v}_1, \overline{v}_2, ..., \overline{v}_k)=\{a_1\overline{v}_1+a_2\overline{v}_2+ ... + a_k\overline{v}_k\text{ }|\text{ }a_1, ..., a_k\in \mathbb{R} \}.$$
Aliavaruuden \(W\) \textbf{dimensio} \(\text{dim}(W)\) on alivaruuden \(W\) kannan vektoreiden lukumäärä.\\
 \\
 \\
\textbf{Esim.} Tutki, kuuluuko vektori \(\overline{w}=(6, 3, 2, -1)\) vektoreiden \(\overline{v}_1=(0, -1, 2, 1)\), \(\overline{v}_2=(2, 0, 1, -1)\) ja \(\overline{v}_3=(4, 2, 2, 0)\) virittämään aliavaruuteen \(\text{span}(\overline{v}_1, \overline{v}_2, \overline{v}_3)\).

\newpage
\section{Vapaus ja kanta}
Vektorijono \((\overline{v}_1, \overline{v}_2, ..., \overline{v}_k)\) on \textbf{vapaa} eli \textbf{lineaarisesti riippumaton}, jos seuraava ehto pätee: jos $$c_1\overline{v}_1+c_2\overline{v}_2 + ... + c_k\overline{v}_k=\overline{0}$$ joillakin \(c_1, ..., c_k \in \mathbb{R}\), niin $$c_1=0, c_2=0, ..., c_k=0.$$
Jos jono ei ole vapaa, niin se on \textbf{sidottu}.\\
 \\
\textbf{Esim.} Olkoot \(\overline{v}_1=(1, 2)\) ja \(\overline{v}_2=(-3, -1)\). Onko jono \((\overline{v}_1, \overline{v}_2)\) vapaa?\\
 \\
  \\
Vektorijono \((\overline{w}_1, \overline{w}_2, ..., \overline{w}_k)\) on aliavaruuden \(W\) \textbf{kanta}, jos $$W=\text{span}(\overline{w}_1, \overline{w}_2, ..., \overline{w}_k)$$ ja jono $$(\overline{w}_1, \overline{w}_2, ..., \overline{w}_k)$$ on vapaa.
 \\
  \\
\textbf{Esim.} Olkoot \(\overline{e}_1=(1, 0)\) ja \(\overline{e}_2=(0, 1)\). Osoita, että jono \((\overline{e}_1, \overline{e}_2)\) on avaruuden \(\mathbb{R}^2\) kanta.

\newpage
\section{Ominaisarvo ja ominaisvektori}
Oletetaan, että \(A\) on \(n\times n\) -neliömatriisi. Reaaliluku \(\lambda\) on matriisin \textbf{ominaisarvo}, jos on olemassa sellainen \textbf{ominaisvektori} \(\overline{v}\in \mathbb{R}^n\), että \(\overline{v}\neq \overline{0}\) ja $$A\overline{v}=\lambda \overline{v}.$$ Siis matriisin \(A\) ominaisvektori on vektori, jolle matriisilla \(A\)
kertominen vastaa reaaliluvulla \(\lambda\) kertomista.\\
 \\
\textbf{Esim.} Matriisilla \(A=\begin{bmatrix} 3 & 1 \\ 1 & 3\end{bmatrix}\) on ominaisvektori \(\begin{bmatrix} 1 \\ 1\end{bmatrix}\). Määritä jokin tätä vastaava ominaisarvo.\\
 \\
Samaa ominaisarvoa voi vastata useampi ominaisvektori. Ominaisarvoa \(\lambda\) vastaava \textbf{ominaisavaruus} on joukko $$V_\lambda =\{\overline{v}\in \mathbb{R}^n | A\overline{v}=\lambda \overline{v}\}.$$
\\
Yhtälö saadaan muotoon \(A\overline{v}=\lambda I\overline{v}\) eli \(A\overline{v}-\lambda I\overline{v}\), josta \((A-\lambda I)\overline{v}=\overline{0}\). Jotta yhtälöllä olisi ratkaisuja \(\overline{v}\neq \overline{0}\), on oltava \(\text{det}(A-\lambda I)=0\). Kirjoittamalla determinantti auki saadaan matriisin \(A\) \textbf{karakteristinen polynomi}.\\
 \\
\textbf{Esim.} Määritä matriisin \(A=\begin{bmatrix} 1 & 2 \\ 3 & 2\end{bmatrix}\) karakteristinen polynomi ja kaikki ominaisarvoa \(4\) vastaavat ominaisvektorit.

\newpage
\section{Piste- ja ristitulo}
Vektoreiden \(\overline{x}\) ja \(\overline{y}\) \textbf{pistetulo} on $$\overline{x}\cdot \overline{y}=x_1 y_1+x_2 y_2+ ... + x_n y_n.$$ Vektorin \textbf{normilla} ja pistetulolla on yhteys $$||\overline{x}||^2=\overline{x}\cdot \overline{x}.$$ Pistetulolle pätee myös $$\overline{x}\cdot \overline{y}=||\overline{x}||||\overline{y}||\cos \theta,$$ missä \(\theta\) on vektorien välinen kulma.\\
 \\
\textbf{Esim.} Määritä vektoria \(\overline{r}=-3\overline{i}+5\overline{j}\) vastaan kohtisuora yksikkövektori.
 \\
  \\
   \\
Olkoot \(\overline{a}\) ja \(\overline{b}\) avaruuden \(\mathbb{R}^3\) vektoreita. Tällöin \textbf{ristitulolle} \(\overline{c}=\overline{a}\times \overline{b}\) pätee $$||\overline{c}||=||\overline{a}||||\overline{b}||\sin \alpha,$$ jossa \(\alpha\) on vektoreiden \(\overline{a}\) ja \(\overline{b}\) välinen kulma. Vektorin suunta määräytyy oikean käden säännön mukaan siten, että \(\overline{a} \perp \overline{c}\) ja \(\overline{b} \perp \overline{c}\). Huomaa, että ristitulo ei ole vaihdannainen eikä liitännäinen.\\
 \\
Ristitulovektorin \(\overline{c}\) pituus on vektoreiden \(\overline{a}\) ja \(\overline{b}\) rajoittaman suunnikkaan pinta-ala.\\
 \\
Vektoreiden \(\overline{a}\), \(\overline{b}\) ja \(\overline{c}\) rajoittaman suuntaissärmiön tilavuus saadaan \textbf{skalaarikolmitulon} \(\overline{a}\times \overline{b}\cdot \overline{c}\) avulla.\\
 \\
\textbf{Esim.} Määritä suuntaissärmiön tilavuus, kun \(\overline{a}=2\overline{i}+\overline{j}\), \(\overline{b}=\overline{i}+2\overline{j}\) sekä \(\overline{c}=3\overline{i}+\overline{k}\).

\newpage
\section{Sovelluksia}
Olkoot \(\overline{v}, \overline{w}\in \mathbb{R}^n\), \(\overline{w}\neq \overline{0}\). Tällöin vektorin \(\overline{v}\) \textbf{projektio} vektorin \(\overline{w}\) virittämälle aliavaruudelle on sellainen vektori \(\overline{p}\in \mathbb{R}^n\), että vektori \(\overline{p}\) on yhdensuuntainen vektorin \(\overline{w}\) kanssa ja vektori \(\overline{v}-\overline{p}\) on kohtisuorassa vektoria \(\overline{w}\) vastaan. Tällöin $$\overline{p}=\text{proj}_{\overline{w}} (\overline{v})=\frac{\overline{v}\cdot\overline{w}}{\overline{w}\cdot \overline{w}}\overline{w}.$$
 \\
\textbf{Esim.} Suora \(S\) kulkee pisteiden \(A=(2, -3, 5)\) ja \(B=(4, 1, 7)\) kautta. Määritä pisteen \(C = (4, -1, 9)\) etäisyys suorasta \(S\) projektion avulla.\\
 \\
 \\
\textbf{Tason normaalimuotoinen yhtälö}\\
Piste \(Q=(x, y, z)\) on tasossa \(T\), jos ja vain jos \(\overline{n}\cdot (\overline{q}-\overline{p})=0\), missä \(\overline{n}\) on jokin tasoa \(T\) vastaan kohtisuora vektori.
\end{document}
