\documentclass[12pt,finnish]{exam}
\usepackage[utf8]{inputenc}
\usepackage{graphicx}
\usepackage[rightcaption]{sidecap}
\usepackage{wrapfig}
\usepackage{babel}
\usepackage{enumitem}
\usepackage{amsmath}
\usepackage{amssymb}
\usepackage[T1]{fontenc}
\usepackage{tikz}
\setlist[enumerate,1]{leftmargin=*}

\begin{document}
 \section*{Matemaattinen fysiikka I}
 
\vspace{5mm}
 
\begin{questions}
\bfseries
\question
\mdseries
Tarkastellaan \(r\)-säteistä ympyrärataa pitkin kulmanopeudella \(\omega\) liikkuvaa kappaletta, johon vaikuttaa voima \(\mathbf{F}=x\mathbf{i}+y\mathbf{j}\). Laske \(\displaystyle \oint_C \mathbf{F}(\mathbf{r})\cdot \text{d}\mathbf{r}\), kun \(C\) on kappaleen rata yhden kierroksen aikana. Mitä tuloksen perusteella voidaan sanoa voiman \(\mathbf{F}\) konservatiivisuudesta?

\bfseries
\question
\mdseries
Pallo potkaistaan ilmaan alkuvauhdilla \(v_0\) potkaisukulman ollessa \(\alpha _0\). Muodosta pallon paikkavektori ja näytä, että pallon mekaaninen energia säilyy ilmalennon aikana. Ilmanvastusta ei oteta huomioon.

\bfseries
\question
\mdseries
Jousessa roikkuva punnus poikkeutetaan alaspäin tasapainoasemastaan. Syntyvää värähtelyä vaimentaa ilmanvastus, joka on suoraan verrannollinen punnuksen nopeuteen verrannollisuuskertoimella \(c=0,\! 4\). Punnuksen massa on \(1,\! 0\text{ kg}\) ja jousen jousivakio \(30,\! 0\text{ N/m}\). Kuinka pitkän ajan kuluttua värähtelyn amplitudi on pienentynyt kymmenesosaan alkuperäisestä?

\bfseries
\question
\mdseries
Kappaleen (massa \(1,\!4\text{ kg}\)) nopeus on \(\mathbf{v}=(t^2-1)\mathbf{i}+(2t+1)\mathbf{j}+t\mathbf{k}\). Kappale on origossa hetkellä \(t=0\). Määritä kappaletta liikuttavan voiman \(\mathbf{F}\) tekemä työ ja sen keskimääräinen teho välillä \(0\leq t\leq 5\).
\end{questions}

\thispagestyle{empty}


\end{document}

